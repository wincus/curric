\begin{rubric}{Cursos Realizados}

\noentry{1996--2006}

\entry*[Año 2012]
	Desarrollo de aplicaciones para plataformas {\bfseries Android}. 20hs

\entry*[Año 2010]
	{\bfseries Los 7 Habitos de las personas altamente efectivas} Taller. 20hs

\entry*[Año 2010]
	{\bfseries JAVA} Programación Java. 140hs 

%\entry*[Año 2008]
%	{\bfseries PMI} Progect Managment Institute Course. 10hs

\entry*[Año 2008]
	{\bfseries CCNP} Cisco Networking Academy "Building Scalable Internetworks". 40hs

\entry*[Año 2008] 
	{\bfseries CCNA1} Cisco Networking Academy "Network Fundamentals".  40hs

\entry*[Año 2008] 
	{\bfseries WIMAX Technologies} Alvarion Manufacturer. 4hs

\entry*[Año 2006] 
	{\bfseries Networking en ambientes GNU/Linux} Universidad Tecnológica Nacional, 
Facultad Regional Mendoza. 30hs

\entry*[Año 2005] 
	{\bfseries IPv6} Universidad de Mendoza, Facultad de Ingeniería. 10hs

\entry*[Año 2004]
	{\bfseries Estructura de Datos} Dictado por la Universidad de Vallaloid a través del programa ODISEAME (UE).

\entry*[Año 2004]
	{\bfseries Protocolo TCP/IP} Dictado por la Universidad de Vallaloid a través del programa ODISEAME (UE).

\entry*[Año 2003]
	{\bfseries Programación orientada a objetos.} Universidad Tecnológica Nacional.

\entry*[Año 2003]
	{\bfseries 3er Exposición de GNU/Linux} Centro de Congresos y exposiciones.

\entry*[Año 2002]
	{\bfseries 2da Exposición de GNU/Linux} Universidad Tecnológica Nacional.

\entry*[Año 2001]
	{\bfseries Curso de GNU/Linux} Universidad Tecnológica Nacional.

\entry*[Año 2000]
	{\bfseries Curso de Microcontroladores PIC} dictado por el Ing. Antonio R. Tafanera.

\end{rubric}
